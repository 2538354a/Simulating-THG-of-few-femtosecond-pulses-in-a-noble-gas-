\documentclass[a4paper]{jpconf}
\usepackage{amsmath}
\usepackage{amssymb}
\usepackage[UKenglish]{babel}
\usepackage{graphicx}
\usepackage{hyperref}
\hypersetup{colorlinks=false, bookmarks=true}
\usepackage{float}
\usepackage{comment}
\usepackage{caption}
\usepackage[skip=0.5ex]{subcaption}


\begin{document}
%\maketitle

\pagenumbering{arabic}
\title{Modelling of nonlinear light up-conversion from intense femtosecond laser pulses}
\author{David Amorim (University of Glasgow)}
\address{DESY Summer Student Programme 2023 \\ Group:  Attosecond Science (CFEL-ATTO) \\ Institute: Centre for Free-Electron Laser Science (CFEL)\\ Supervisor: Josina Hahne}

\thispagestyle{plain}
\pagestyle{plain}
\setlength{\footskip}{5pt}

\begin{abstract}
XXX
\end{abstract}


\section{Introduction}
The electronic motion associated with chemical reactions or atomic energy transitions takes place on femotsecond ($1 \text{fs} = 10^{-15} \text{s}$) to attosecond ($1 \text{as} = 10^{-18} \text{s}$) time scales. Time-resolved imaging of molecular dynamics therefore requires ultrashort laser pulses in the few-femtoseconds regime. An important type of electronic motion is that excited by the absorption of ultraviolet (UV) radiation: a variety of biochemical processes, such as DNA damage, are known to be caused by UV excitation. Generating few-femtosecond UV pulses for time-resolved measurements of UV-excited molecules plays a crucial part in further understanding and eventually controlling these effects. \par 
The CFEL-ATTO group at DESY produces ultrashort UV pulses in the femtosecond regime via third-harmonic generation (THG) in a gas cell \cite{galli2019}. This process is sensitive to a large range of parameters which affect the duration and energy of the generated UV pulse. Numerical simulations of the interactions between the laser pulse and the gas can therefore be advantageous to better understand how THG takes place and how it is affected by the experimental conditions. \par 
This report summarises the work carried out as part of a DESY Summer Student project focussed on producing simulations of the THG process. The simulations were mainly carried out using the propagation solver \textit{Luna.jl} (\cite{brahms2023}) while the \textit{COMSOL Multiphysics} software was used to model the gas density distribution. The aim of the simulations was to reproduce the experimental conditions in the gas, in order to study which effects dominate the THG process and the shaping of the UV pulse. Further, parameter scans were simulated to investigate how changes of different variables in the experimental set-up affect the output pulse. \par 
------------------------------------------------
Mention:
\begin{itemize}
\item what is covered in which section
\item briefly list findings (main relevant processes in gas, experimental conditions...)

\end{itemize}
-----------------------------------------------

\section{Theoretical Background}
This section gives a brief overview on key properties of ultrashort (femtosecond) laser pulses, the differential equation governing their propagation, and THG as well as other relevant nonlinear effects. An in-depth treatment of these topics can be found in textbooks like \cite{keller2021, new2011}. 

\subsection{Ultrashort Laser Pulses}
An ultrashort laser pulse can be described as a carrier wave at carrier frequency $\omega_0$ with an amplitude modulated by some envelope function. In the femtosecond domain the envelope may only contain a few cycles of the carrier wave. Using the analytic representation of the electric field vector the time-domain representation of an ultrashort laser pulse can be written as 
\begin{equation}
\mathbf{E}(t) = A(t) e^{i[ \omega_o t + \phi(t)]} \hat{\mathbf{u}}_p,
\end{equation}
where $A(t)$ represents the envelope function, $\phi(t)$ is the temporal phase and $\hat{\mathbf{u}}_p$ is a unit vector indicating the polarisation of the field. Spatial dependencies are left implicit for simplicity. Similarly, the frequency-domain representation is 
\begin{equation}
\mathbf{E}(\omega) = A'(\omega) e^{i \varphi(\omega)} \hat{\mathbf{u}}_p,
\end{equation}
where $\varphi(\omega)$ is the spectral phase. Note that the effect of the carrier term in the frequency domain is merely a shift and can be thus be absorbed in the definition of $\omega$. \par 
The spectral and temporal phase are often expanded as Taylor-series, with the notation $\phi_n \equiv \frac{\partial^n \phi}{\partial t^n}$, $\varphi_n \equiv \frac{\partial^n \varphi}{\partial t^n}$ used to denote the series coefficients. The zeroth-order terms of both expansions are the same, $\phi_0 = \varphi_0$, and correspond to the carrier-envelope phase (CEP), also known as the absolute phase. The CEP phase desribes the phase difference between the carrier and the envelope. A laser pulse with no terms in $\varphi(\omega)$ higher than $\varphi_0$ is said to be transform-limited. A pulse is said to be chirped if there are nonlinear terms in $\phi(t)$ or $\varphi(\omega)$. \par 
The presence of chirp can result in both spectral and temporal broadening of the pulse envelope. A convenient measure of chirp is therefore the so-called time-bandwidth product $\Delta \omega \Delta t$, where $\Delta \omega$ is the full-width half-maximum (FWHM) of $A'(\omega)$ and $\Delta t$ the FWHM of $A(t)$. It can be shown that the time-bandwidth product is minimised for transform-limited, i.e. unchirped, pulses. In other words, at a given spectral bandwidth the shortest pulse durations are associated with transform-limited pulses, for which case the relation $\Delta \omega \propto (\Delta t)^{-1}$ holds.   


-----------------
\begin{itemize}
\item more on chirp! how does it lead to broadening.... positive versus negative chirp...
\item more on dispersion; also inlcude absorption constant alpha 
\end{itemize}

\subsection{Nonlinear Optics and THG}
For most materials, the polarisation response $\mathbf{P}(t)$ to a moderate electric field $\mathbf{E}(t)$ is linear: $\mathbf{P}(t) = \epsilon_0 \chi \mathbf{E}(t)$, where the linear susceptibility $\chi$ is material-specific and generally frequency-dependent. Stronger field amplitudes result in a nonlinear polarisation response, which can be pertubatively expanded in $\mathbf{E}(t)$:
\begin{equation}\label{eq:P(E)_expansion}
\mathbf{P}(t) = \epsilon_0 \left( \chi^{(1)}|\mathbf{E}(t)| + \chi^{(2)}|\mathbf{E}(t)|^2 +  \chi^{(3)}|\mathbf{E}(t)|^3 + ...  \right) \frac{\mathbf{E}(t)}{|\mathbf{E}(t)|}.
\end{equation}
In the above, $\chi^{(n)}$ corresponds to the $n$-th order electric susceptibility. For simplicity, equation \eqref{eq:P(E)_expansion} leaves the spatial dependence of the polarisation and the driving field implicit. It further treats the $\chi^{(n)}$, which generally are frequency-dependent tensors, as constant scalars. This approximation is valid for isotropic media such as the noble gases considered here [!!!?? SOURCE]. Note that since gases are centrosymmetric all even-order susceptibilities necessarily vanish, e.g. $\chi^{(2)}=0$, so that the third-order response becomes the dominant nonlinear term. Due to the small magnitudes of $\chi^{(5)}$ and the other higher-order odd susceptibilities [INCLUDE TABLE OF VALUES!], in the following only the linear and third-order terms of the expansion $\mathbf{P}(\mathbf{E}(t))$ will be considered. \par 
It is evident from \eqref{eq:P(E)_expansion} how an oscillating electric field in a nonlinear medium results in the generation of co-propagating fields with new frequency components: consider $\mathbf{E}^{(\omega)}(t) = \tilde{\mathbf{E}} \cos(\omega t)$, where all spatial dependence and phase terms have been absorbed into the complex amplitude  $\tilde{\mathbf{E}}$. The resulting polarisation response, due to the linear and third-order term, is then given by 
\begin{equation}\label{eq:int_dep_n}
\mathbf{P}(t) = \epsilon_0 \left( \chi^{(1)} \cos(\omega t) \tilde{\mathbf{E}} + \chi^{(3)} \cos(\omega t)^3 |\tilde{\mathbf{E}}|^2 \tilde{\mathbf{E}}  \right) = \epsilon_0 \left( \left[ \chi^{(1)} + \frac{3}{4} \chi^{(3)} |\tilde{\mathbf{E}}|^2\right] \cos(\omega t) + \frac{1}{4} \chi^{(3)} \cos(3 \omega t) \right)  \tilde{\mathbf{E}}
\end{equation}
Thus, the effect of the third-order susceptibility is two-fold: a field contribution $\mathbf{P}^{(3\omega)}(t)$ oscillating at $3\omega$ is generated and an additional term in $\chi^{(3)} |\tilde{\mathbf{E}}|^2$ is added to the polarisation response  $\mathbf{P}^{(\omega)}(t)$ at the fundamental frequency. The former process can be identified as THG: a driving electric field oscillating at some frequency will generate a frequency-tripled co-propagating field due to third-order nonlinear interactions in the medium. This is one instance of a range of related nonlinear effects which result in the generation of a higher-frequency field component, known as frequency up-conversion. \par 
Note that a third-harmonic contribution is generated in-phase and co-propagating with the driving wave at each point of the nonlinear medium. Due to dispersion, $k=k(\omega)$, the third-harmonic and the fundamental will generally experience different phase shifts on propagation, resulting in a phase mismatch $\Delta k = k(3 \omega)- 3k (\omega)$ between the different third-harmonic components generated at different points along the axis of propagation. This can result in destructive interference of the individual $3\omega$ components. Thus, the phase-mismatch parameter $\Delta k$, or, equivalently, the GVD [DEFINE THIS], are important determinants of the achievable THG conversion efficiency $\eta_{THG}$, a measure of the energy of the generated third-harmonic relative to the input pulse energy.  \par 
The latter process [REWRITE THIS, REFERENCE TOO FAR BACK] is equivalent to an intensity-dependent contribution being added to the refractive index of the medium. The refractive index, $n$, of a linear medium is commonly defined as $n = \sqrt{1+ \mathbf{P}(t)/[\epsilon_0 \mathbf{E}(t)]} = \sqrt{1 + \chi}$. Analogously, the effective refractive index $n_{\text{eff}}$ in a nonlinear medium can be defined as
\begin{equation}
n = \sqrt{1+ \frac{\mathbf{P^{(\omega)}}(t)}{\epsilon_0 \mathbf{E^{(\omega)}}(t)}},
\end{equation}
where $\mathbf{P}^{(\omega)}(t)$ is the polarisation response to $\mathbf{E}^{(\omega)}(t)$ oscillating at the fundamental frequency. Due to the presence of the third-order term, and taking into account that $\chi^{(1)} \gg \frac{3}{4} \chi^{(3)} |\tilde{\mathbf{E}}|^2$ :
\begin{equation}
n_\text{eff} = \sqrt{1 + \chi^{(1)} + \frac{3}{4} \chi^{(3)} |\tilde{\mathbf{E}}|^2} \approx n + \frac{3}{8n} \chi^{(3)} |\tilde{\mathbf{E}}|^2 \equiv n + \Delta n I = n_0 + n_2 I,
\end{equation}
where $I = |\tilde{\mathbf{E}}|^2/[2Z_0]$ is the intensity of the driving field, $Z_0$ the impedance of the wave in the medium, and $\Delta n$ is the intensity-dependent change in effective refractive index. \par 
This intensity-dependent refractive index has important implications for the propagation of laser pulses in nonlinear media. An important consequence is self-phase modulation (SPM): since the intensity of a beam is significantly higher at its peak than towards the wings, the centre of the beam will experience a higher effective refractive index, and therefore phase-shift on propagation, than off-peak components. This can result in spectral broadening of the pulse [MORE DETAIL???] [IN TIME DOMAIN;]. The intensity-dependent refractive index is also related to effects like self-focussing [SAME THING BUT SPATIAL]. \par 
Note that the umbrella term `Kerr effects' is commonly used to refer to processes associated with $\chi^{(3)}$, including THG, and this practice will be adopted here. \par 
For the present context, there is a further relevant group of nonlinear effects that is highly non-perturbative and therefore cannot be captured by the expansion \eqref{eq:P(E)_expansion}. These are photo-ionisation effects: at sufficiently high field amplitudes, the atoms or molecules in the nonlinear medium become ionised, which affects their interaction with the driving field and generated third-harmonic. Particularly the values of the $\chi^{(n)}$ and the absorption coefficients are affected. The effects of ionisation on the polarisation response of a quasistatic gas cell are captured by the ADK model. While ionisation limits the achievable third-harmonic pulse energy due to absorption, it affects the spatio-temporal pulse profile and can result in spectral broadening, as is discussed later. \par 
The total polarisation response $\mathbf{P}$ in a nonlinear medium can thus be split up into a linear and non-linear part, $\mathbf{P} = \mathbf{P}^L + \mathbf{P}^{NL}$, where the latter can further be subdivided into Kerr and ionisation effects: $\mathbf{P}^{NL} = \mathbf{P}^{\text{Kerr}} + \mathbf{P}^{\text{ion}}.$ MENTION HOW LATER THIS IS IMPLEMENTED IN THE UPPE?


MAKE SURE THAT IT IS CLEAR THAT ALL MATERIALS ARE NONLINEAR AS LONG AS E IS HIGH ENOUGH!!

SAY SOMEWHERE THAT FOCUSSING ON GAS!! BC OF LOW GVD 


ALSO MENTION THAT $\chi$ IS PRORTIONAL TO MOLAR SUSCEPTIBLITY, HENCE INCREASING DENSITY INCREASES CHI


NOTE: since intensity highest at centre of beam, THG efficiency is more efficient there, so that the THG pulse is narrower, both in space and in time, than the input pulse. Temporally, this is exepcted to by by $\sim \sqrt{3}$ shorter. 


USE n0, n2 notation?


SELF-STEEPENING: due to IDRI, peak of pulse experiences higher refractive index and hence (assuming normal dispersion) a slower group velocity than the wings. This results in the leading edge (or trailing edge, for anomalous dispersion) getting steeper. Self-steepening is associated with spectral broadening ?!. 

\subsection{Ultrashort Laser Pulses}

The effects of $\phi_1$ and $\varphi_1$ are spectral and temporal shifts, respectively. Further, $\phi_1$ is used to define the instantaneous angular frequency 
\begin{equation}
\omega_\text{inst}(t) = \omega - \phi_1,
\end{equation}  
while $\varphi_1$ is used to define the group delay $t_\text{group}(\omega) = \varphi_1$. \par 
The term $\phi_2$ corresponds to quadratic variations in the temporal phase and hence to a linear variation in the instantaneous frequency with time. This is known as a linear chirp. If the pulse is positively chirped, the frequency increases with time while in a negatively chirped pulse the frequency decreases. Thus, the leading edge of a positively chirped pulse is of a lower frequency than the trailing edge and vice-versa for negatively chirped pulses.  In regions of normal dispersion this results in a positively chirped pulse being extended in time and negatively chirped pulses being compressed. The reverse hold in regions of anomalous dispersion. IS $\varphi_2$ the same as $k_2$, i.e. GVD?? APPARENTLY THE SAME BUT THE K2 IS PER UNIT LENGTH. Pulse elongation of TL pulse due to $\varphi_2$:
$\tau_{out} = \tau_{in} \sqrt{1+ \left( \frac{4 \ln 2 \varphi_2}{\tau_{in}^2} \right)^2}$. When DOES THIS HOLD? WHAT ABOUT NEGATIVE??

\par 
Thus, nonlinear variations of the temporal phase result in spectral broadening by inducing chirp, which may also affect the pulse duration, if dispersion is present. One such nonlinear variation of the phase in time is given by effects of the intensity-dependent refractive index (IDRI), which results in a larger nonlinear phase increase at the (temporal) centre of the pulse and hence in chirp and spectral broadening if the pulse was initially unchirped. May also lead to spectral compression if the initial pulse was chirped oppositely.  \par 

Spectral phase is affected by spatial propagation: $\varphi_out (\omega) = \varphi_in(\omega) + \varphi_{prop}(\omega)$ [note: the effect on the phase in the time domain is not as simple! cannot simply add the phases!!]. The phase due to propagation of lenght $L$ in a medium is $\varphi_{prop} =k(\omega) L$. We can expand $k$ as a Taylor series... [second term is GVD!]. Multiplying GVD by L gives GDD [group delay dispersion]. Generally, GVD/GDD is positive, resulting in elongated pulses on propagation. NOTE: chirp of the pulse can be corrected/modified by passing it through an appropriate medium to cancel out the phase terms. \par 


\paragraph{GVD and GDD}
When a spectral phase arises due to passage through a dispersive medium of length $L$, we can write $\varphi(\omega) = - \beta(\omega) L$, where $\beta$ (often: $k$) is the propagation constant in the medium. We often write this in terms of Taylor expansion [about the carrier frequency $\omega_0$]s:
$$ \varphi(\omega) = \varphi_0 + \omega \varphi_1 + \omega^2 \frac{\phi_2}{2} + ...$$
and the same for $\beta_n$ with $\varphi_n = - \beta_n L$. \par 
The carrier term propagated at the phase velocity $v_p = \omega_0/\beta_0$ and is unaffected by the variation of $\beta$ or $\varphi$ with $\omega$. The envelope function, however, is affected by the form of $\beta(\omega)$. When $\beta$ is linear, $\beta(\omega) = \beta_0 + \beta_1 (\omega - \omega_0)$ then the output pulse is an undistorted replica of thei nput pulse, which travels with group velocity $v_g = \beta_1^{-1}$. When $beta$ is not linear, the shape of the envelope is changed during propagation, which is known as dispersion. \par 
Note that $\beta(\omega) = \omega n(\omega)/c$

-----------------------------------------------------------

mention how high peak intensities are sufficient in ultrashort pulses to make nonlinear effects relevant 

mention bandwith theorem


\subsection{The Unidirectional Pulse Propagation Equation and \emph{Luna} }
There is a variety of differential equations that can be used to describe the propagation of an ultrashort laser pulse in a waveguide or free space. As is shown in [ref], most of these can be derived from the z-propagated unidirectional pulse propagation equation (UPPE). Several variations exist but its general form for a pulse of frequency $\omega$ travelling in the $z$-direction is 
\begin{equation}
\frac{\partial}{\partial z} \mathbf{E}(\omega, \mathbf{k}_\perp, z)= \mathcal{L}(\omega, \mathbf{k}_\perp, z)\mathbf{E}(\omega, \mathbf{k}_\perp, z) + \frac{i \omega}{N_{\text{nl}}} \mathbf{P}_\text{nl}(\omega, \mathbf{k}_\perp, z),
\end{equation}
where $\mathbf{E}(\omega, \mathbf{k}_\perp, z)$ is reciprocal-space electric field amplitude in the frequency domain, $\mathbf{k}_\perp$ is a generalised transverse wavevector,  $\mathcal{L}(\omega, \mathbf{k}_\perp, z)$ is a linear operator describing dispersion, diffraction, and absorption, while $\mathbf{P}_\text{nl}$ is the nonlinear polarisation with normalisation factor $N_\text{nl}$. \par 
The nonlinear polarisation response to a given electric field is computed most easily in real space and the time domain: $\mathbf{P}(t, \mathbf{r}_\perp,z) = \mathcal{P}_\text{nl}(\mathbf{E}(t,\mathbf{r}_\perp,z))$, where $t$ is time, $\mathcal{P}_\text{nl}$ is an appropriate operator [RELATE TO Pnl ABOVE?!] and $\mathbf{r}_\perp = (r, \theta)$ represents the transverse spatial coordinate. The real-space-time domain representation of $\mathbf{P}_\text{nl}$ can then be related to the reciprocal-space-frequency domain version via a temporal Fourier transform, $\mathcal{F}$, as well as an appropriate transverse spatial transform, $\mathcal{T}_\perp$:
\begin{equation}
\mathbf{P}(\omega, \mathbf{k}_\perp,z) = \mathcal{F}\left[ \mathcal{T}_\perp \left[ \mathbf{P}(t, \mathbf{r}_\perp,z) \right](t, \mathbf{k}_\perp, z)  \right]. 
\end{equation}
Note that the real-space-time domain field $\mathbf{E}(t, \mathbf{r}_\perp, z)$ can be recovered from $\mathbf{E}(\omega, \mathbf{k}_\perp, z)$ via the appropriate inverse transformations $\mathcal{F}^{-1}$ and $\mathcal{T}^{-1}_\perp$. \par 

Hence, once the operators $\mathcal{L}$ and $\mathcal{P}_\text{nl}$ as well as appropriate definitions of the transverse (reciprocal) space $\mathbf{k}_\perp$ and associated transform $\mathcal{T}_\perp$ are known, the UPPE can be solved as an initial value problem in $z$, using appropriate cycles of transforms at each step. \par 
The choice of $\mathbf{k}_\perp$ and $\mathcal{T}_\perp$ is mostly a matter of the modal expansion of the field in terms of a set of orthogonal spatial modes [the choice of the transform also affects the shape of the linear operator; Note: UPPE derived from Maxwell's equation by substituting a modal expansion!]. These may be waveguide modes or, in the present case, free space modes. The frequency-domain field can be expanded as 
\begin{equation}
\mathbf{E}(\omega, \mathbf{r}_\perp, z) = \sum_j \tilde{E}_j (\omega,z)  \hat{\mathbf{e}}_j (\mathbf{r}_\perp, z),
\end{equation}
where $\hat{\mathbf{e}}_j (\mathbf{r}_\perp, z)$ is the orthonormal transverse field distribution of the $j$th mode and $\tilde{E}_j (\omega,z)$ is the frequency-domain field in mode $j$. The modes are normalised via $N_{j, \text{nl}}$ to ensure that $|\tilde{E}(\omega,z)_j|^2$ represents the spectral power density in the $j$th mode. \par 
Given an input field $\mathbf{E}(t, \mathbf{r}_\perp, z)$ the modal decomposition can be calculated via 
\begin{equation}
\tilde{E}_j (\omega,z) = \mathcal{F}\left[ \int_S \hat{\mathbf{e}}_j(\mathbf{r}_\perp,z) \cdot \mathbf{E}(t, \mathbf{r}_\perp, z) d^2 \mathbf{r}_\perp \right],
\end{equation}
where $S$ is the cross-sectional area of the gas cell. Modal decomposition can be used to solve  a scalar version of the UPPE for each $\tilde{E}_j(\omega,z)$:
\begin{equation}
\frac{\partial}{\partial z} \tilde{E}_j(\omega,z) = \mathcal{L}_j (\omega,z) \tilde{E}_j(\omega,z)+ i\frac{\omega}{N_{j, \text{nl}}} \tilde{P}_{j,\text{nl}},
\end{equation}
where $\tilde{P}_{j,\text{nl}}$ represents the projection of the polarisation onto the $j$th mode:
\begin{equation}
\tilde{P}_{j, \text{nl}}(\omega, z) =\mathcal{F}\left[ \int_S \hat{\mathbf{e}}_j(\mathbf{r}_\perp,z) \cdot \mathbf{P}_{\text{nl}}(t, \mathbf{r}_\perp, z) d^2 \mathbf{r}_\perp \right].
\end{equation}
The linear operator acting on the $j$th mode can be described as 
\begin{equation}
\mathcal{L}_j(\omega, z) = i \left( \frac{\omega}{c} n_{\text{eff}}(\omega,z) - \frac{\omega}{v} \right),
\end{equation}
where $n_\text{eff}$ is the complex effective index (whose real part represents the effective refractive index [not taking into account IDRI?!] and whose imaginary part gives the attenuation coefficient), $c$ the speed of light in vacuum, and $v$ is the velocity of an arbitrary reference frame. Here, $v$ is taken to be the group velocity at the central wavelength $\lambda_0$. \par 
For the present case of pulse propagation in free-space (`free-space' as opposed to in waveguide or fibre; does propagate through gas, not vacuum) the appropriate reciprocal transform $\mathcal{T}_\perp$ is the (zeroth-order) Hankel transform, defined by
\begin{equation}
\mathcal{H} \equiv \int_0^\infty dr J_0 (kr)r
\end{equation},
where $J_0$ is the zeroth-order Bessel function. DEFINE THIS IN APPLICATION TO E OR P. Thus, $\mathbf{r}_\perp = (r, \theta)$ and $\mathbf{k}_\perp = (k, \theta)$ where any theta-dependence can be dropped due to the cylindrical symmetry. 
...
Note that this assumes radial symmetry about the propagation axis, which is justified due to the relevant gemoetry of the gas cell. THUS, ONE-DIMENSIONAL TRANSFORM ONLY IN r?!

-----------------------------------------------------------------------------------------------------
The total nonlinear polarisation can be calculated from the total electric field while separating Kerr and ionisation contributions (ignoring higher-order terms of \eqref{eq:P(E)_expansion}):
\begin{equation}
\mathbf{P}_\textbf{nl} (t, \mathbf{r}_\perp, z) = \mathbf{P}_\text{Kerr}(\mathbf{E}) + \mathbf{P}_\text{ion} (\mathbf{E}).
\end{equation}
Assuming that $\chi^{(3)}$ is not dependent of frequency, and hence that THG is instantaneous, as well as approximating the third-order susceptibility as a scalar due to the isotropy of the gas, the Kerr term can be written as 
\begin{equation}
\mathbf{P}_\text{Kerr}(t, \mathbf{r}_\perp, z) = \epsilon_0 \chi^{(3)} |\mathbf{E}(t, \mathbf{r}_\perp, z)|^2\mathbf{E}(t, \mathbf{r}_\perp, z).
\end{equation}
At sufficiently high intensities, photoionisation effects set in. This affects pulse propagation as the freed electrons are polarisable and as energy is lost to the ionisation potential. [NOTE; the following two things should be $n_0$ and $n_e$] If $a_0$ is the original number of neutral atoms and $a_e(t, \mathbf{r}_\perp, z)$ the number of free electrons at a given point in time and space then $a_e$ can be calculated via
\begin{equation}
\frac{\partial}{\partial t} a_e = (a_0 - a_e) W(\mathbf{E})
\end{equation}
where $W(\mathbf{E})$ is the ionisation rate. The ionisation rate can be found via THE ADK MODEL?? [link papers]. Given $a_e$, the induced nonlinear polarisation due to ionisation can be shown to be 
\begin{equation}
\frac{\partial}{\partial t} \mathbf{P}_\text{Ion}(t, \mathbf{r}_\perp, z) = \mathbf{E}(t, \mathbf{r}_\perp, z) \frac{I_p}{|\mathbf{E}(t, \mathbf{r}_\perp, z)|^2} \frac{\partial}{\partial t} a_e (t, \mathbf{r}_\perp, z ) + \frac{e^2}{m_e^2} \int_{-\infty}^t a_e (t', \mathbf{r}_\perp, z) \mathbf{E}(t', \mathbf{r}_\perp, z) dt',
\end{equation}
where $I_p$ is the ionisation energy, $e$ and $m_e$ are the electronic charge and mass.  
-------------------------------------------------------------------------------------------------------------------------


---------------------------------------------------------------------------------------------
NOTE: small e should be changed to mathcal E or something to avoid confusion! not basis vector but field! orthogonal with respect to int dr, not dot product!
$\int_S \hat{\mathbf{e}_j} \cdot \hat{\mathbf{e}}_k d^2 \mathbf{r}_\perp = \delta_{jk}$

DUE TO CYLINDRICAL SYMMETRY: expanding on Bessel function basis, not plane waves? Apart from that, free-space transverse modes?

I GUESS CYLINDRICAL WAVEGUIDE MODES? BESSEL FUNCTIONS...
		-> free space modes?!
		
		IMPORTANCE OF SPATIAL TRANSFORM: comes in during UPPE derivatioin?

THERE MIGHT ACTUALLY NOT BE A RELATION BETWEEN MODAL EXPANSION AND HANKEL TRANSFORM??


RELATE UPPE TO MAXWELL'S EQUATIONS!
		-> UPPE CAN BE DERIVED FROM MAXWELL based on a modal expansion  -> thus slightly different forms for different contexts [link paper]

MENTION LUNA!!!

WRITE ABOUT Pkerr, Pion ,....

MENTION HANKEL TRANSFORM! -> find out more! which order zeroth order?? which spatial modes does this correspond to? bessel? -> one dimensional Hankel transform essentially?

HOW TO FIND THE SPATIAL MODE VECTORS???
	MAYBE VECTORS NOT REALLY SPATIAL!??

POTENTIALLY START WITH THIS SECTION?

Hankel transofmr is used instead of eq 9 for modal decomposition! Hankel transform essentially 2D Fourier transform for radially symmetric function

STILL PLANE WAVES!!


two terms in the photoionisation response: loss term (first) and phase modulation term. This is why ionisation can be helpful due to spatiotemporal re-shaping.

\section{Experimental Conditions}
The CFEL-ATTO group generates few-femtosecond UV laser pulses via THG in a gas cell. This is driven by a pulsed near-infrared (NIR) Ti:sapphire laser with a central wavelength of 750nm and a repetition rate of 1kHz. Various pulse-shaping and compression techniques are used to produce NIR pulses with a duration of 5-6fs and a pulse energy of XX$\mu$J. A (variable) fraction of the pump beam is then linearly polarised and focussed into a vacuum chamber for UV generation, resulting in a spatially approximately Gaussian beam with a beam waist of 65$\mu$m at the focus. The chamber contains a fused silica cell, a gas nozzle and a pumping system. Its specifications are central to the THG process and are described in more detail in the following. \par 
The fused silica cell is equipped with a laser machined central 3mm-long channel with a 0.5mm diameter. Pressurised gas is pumped through the channel, with a two-stage differential pumping system on both sides of the central cell ensuring a tight confinement of the gas in this region. As the pump beam passes through the cell and interacts with the pressurised gas, THG takes place which results in UV pulses being generated. The UV components are then separated from the pump beam upon exiting the vacuum chamber using dispersive mirrors. 

-----------------------------
\begin{itemize}
\item include figure?!!!
\end{itemize}



\section{Simulation Results}

The simulations were carried out with \emph{Luna}, as described above. The simulation parameters were set up to match the experimental parameters are closely as possible. \par 
 The gas cell was modelled as a 3mm long radially symmetric chamber with a 0.5mm diameter. The spatial profile of the input beam was taken to be a Gaussian with a 65$\mu$m waist size focussed at the centre of the cell, based on measurements of the spatial beam profile. The input NIR pulse was fed in based on spectral measurements (FROG) of its temporal profile [SENTENCE MAKES NO SENSE]. The central wavelength was taken to be 730nm, based on NIR input pulse measurements, whereas the pulse length of the input pulse was 5.9fs. The pressure at the cell edges was taken to be 1mbar. The total wavelength window considered ranged from 100nm to 1000nm, with the range 100-360nm considered as the UV component and 600-1000nm as the IR component. The nonlinear effects considered where the Kerr effect as well as ADK ionisation, although the latter was turned off in some cases, in order to isolate the effects of ionisation. The CEO phase of the input beam was set to zero without measurement, as simulations showed little impact of the CEO phase on the simulation output, since the input pulse involves sufficient cycles. \par 
Note that two different models for the gas density profile were considered. The first one considered only the cell itself, with a simple density gradient constructed based on specifying the central pressure as well as the edge pressure. The second model was more sophisticated and based on simulations produced with the COMSOL Multiphysics software. It took into account not only the gas cell itself but also the geometry of the 7mm glass chip surrounding it, as well as the effects of the differential pumping system. \par 
The simulations were focussed on investigating parameters that could be experimentally controlled. These were the gas, the central gas pressure, and the input beam power. The gases considered were Argon, Neon, Helium, Krypton, Xenon, Nitrogen, and Nitrous Oxide (N$_2$O). The beam powers considered were 75mW, 150mW, 200mW, 300mW, and 400mW. The simulated central pressures ranged from 0.1bar to 5.1 bar in 0.1bar increments. \par 
The simulation output included the UV spectrum at the end of the cell (or chip), the total output UV energy, the THG conversion efficiency, the pulse duration of the UV pulse, and the cell (or chip) position at which the UV energy intensity peaked. Additionally, the option to visualise the results in more detail was included, with plots such as: time-domain IR pulse profile, time-domain UV pulse profile, spatiotemporal UV beam evolution, spatiotemporal IR beam evolution, on-axis frequency evolution, UV spectral evolution, on- and off-axis UV and IR intensity profiles, UV and IR energy profiles along z, various spectra. \par 


NOTE: ADK MODEL DOES NOT INVOLVE MULTIPHOTON IONISATION BUT IS COMPUTATIONALLY LESS EXPENSIVE
-> inaccurate at low pressures! 
\section{Experimental Results}

\section{Discussion}
Comparison with results at times difficult: pressure and energy of output beam hard to measure


Many more factors to be investigated (beam focus position, beam size, higher beam powers...)


Improve simulations by improving COMSOL gas density model (need to use high Mach number model instead, current model does not take into account gas interaction). 

\section{Conclusions}

\section*{Acknowledgments}
I want to thank Josina Hahne for the supervision of the project and Vincent Wanie for his guidance and support. I also want to thank Chris Brahms for providing help with \textit{Luna.jl} as well as Olaf Behnke and Andreas Przystawik for organising the DESY Summer Student programme.  


\section*{References}
\bibliographystyle{osa}
\bibliography{refs.bib}

\end{document}