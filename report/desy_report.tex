\documentclass[a4paper]{jpconf}
\usepackage{amsmath}
\usepackage{amssymb}
\usepackage[UKenglish]{babel}
\usepackage{graphicx}
\usepackage{hyperref}
\hypersetup{colorlinks=false, bookmarks=true}
\usepackage{float}
\usepackage{comment}
\usepackage{caption}
\usepackage[skip=0.5ex]{subcaption}


\begin{document}
%\maketitle

\pagenumbering{arabic}
\title{Modelling of nonlinear light up-conversion from intense femtosecond laser pulses}
\author{David Amorim (University of Glasgow)}
\address{DESY Summer Student Programme 2023 \\ Group:  Attosecond Science (CFEL-ATTO) \\ Institute: Centre for Free-Electron Laser Science (CFEL)\\ Supervisor: Josina Hahne}

\thispagestyle{plain}
\pagestyle{plain}
\setlength{\footskip}{5pt}

\begin{abstract}
XXX
\end{abstract}


\section{Introduction}
The electronic motion associated with chemical reactions or atomic energy transitions takes place on femotsecond ($1 \text{fs} = 10^{-15} \text{s}$) to attosecond ($1 \text{as} = 10^{-18} \text{s}$) time scales. Time-resolved imaging of molecular dynamics therefore requires ultrashort laser pulses in the few-femtoseconds regime. An important type of electronic motion is that excited by the absorption of ultraviolet (UV) radiation: a variety of biochemical processes, such as DNA damage, are known to be caused by UV excitation. Generating few-femtosecond UV pulses for time-resolved measurements of UV-excited molecules plays a crucial part in further understanding and eventually controlling these effects. \par 
The CFEL-ATTO group at DESY produces ultrashort UV pulses in the femtosecond regime via third-harmonic generation (THG) in a gas cell \cite{galli2019}. This process is sensitive to a large range of parameters which affect the duration and energy of the generated UV pulse. Numerical simulations of the interactions between the laser pulse and the gas can therefore be advantageous to better understand how THG takes place and how it is affected by the experimental conditions. \par 
This report summarises the work carried out as part of a DESY Summer Student project focussed on producing simulations of the THG process. The simulations were mainly carried out using the propagation solver \textit{Luna.jl} (\cite{brahms2023}) while the \textit{COMSOL Multiphysics} software was used to model the gas density distribution. The aim of the simulations was to reproduce the experimental conditions in the gas, in order to study which effects dominate the THG process and the shaping of the UV pulse. Further, parameter scans were simulated to investigate how changes of different variables in the experimental set-up affect the output pulse. \par 
------------------------------------------------
Mention:
\begin{itemize}
\item what is covered in which section
\item briefly list findings (main relevant processes in gas, experimental conditions...)

\end{itemize}
-----------------------------------------------

\section{Theoretical Background}
This section gives a brief overview on key properties of ultrashort (femtosecond) laser pulses, the differential equation governing their propagation, and THG as well as other relevant nonlinear effects. An in-depth treatment of these topics can be found in textbooks like \cite{keller2021, new2011}. 

\subsection{Ultrashort Laser Pulses}
An ultrashort laser pulse can be described as a carrier wave at carrier frequency $\omega_0$ with an amplitude modulated by some envelope function. In the femtosecond domain the envelope may only contain a few cycles of the carrier wave. Using the analytic representation of the electric field vector the time-domain representation of an ultrashort laser pulse can be written as 
\begin{equation}
\mathbf{E}(t) = A(t) e^{i[ \omega_o t + \phi(t)]} \hat{\mathbf{u}}_p,
\end{equation}
where $A(t)$ represents the envelope function, $\phi(t)$ is the temporal phase and $\hat{\mathbf{u}}_p$ is a unit vector indicating the polarisation of the field. Spatial dependencies are left implicit for simplicity. Similarly, the frequency-domain representation is 
\begin{equation}
\mathbf{E}(\omega) = A'(\omega) e^{i \varphi(\omega)} \hat{\mathbf{u}}_p,
\end{equation}
where $\varphi(\omega)$ is the spectral phase. Note that the effect of the carrier term in the frequency domain is merely a shift along the frequency axis and can be thus be absorbed in the definition of $\omega$. \par 
The spectral and temporal phase are often expanded as Taylor-series, with the notation $\phi_n \equiv \frac{\partial^n \phi}{\partial t^n}$, $\varphi_n \equiv \frac{\partial^n \varphi}{\partial t^n}$ used to denote the series coefficients. The zeroth-order terms of both expansions are the same, $\phi_0 = \varphi_0$, and correspond to the carrier-envelope phase (CEP), also known as the absolute phase. The CEP phase desribes the phase difference between the carrier and the envelope. A laser pulse with no terms in $\varphi(\omega)$ higher than $\varphi_0$ is said to be transform-limited. A pulse is said to be chirped if there are nonlinear terms in $\phi(t)$ or $\varphi(\omega)$. \par 
The presence of chirp can result in both spectral and temporal broadening of the pulse envelope. A convenient measure of chirp is therefore the so-called time-bandwidth product $\Delta \omega \Delta t$, where $\Delta \omega$ is the full-width half-maximum (FWHM) of $A'(\omega)$ and $\Delta t$ the FWHM of $A(t)$. It can be shown that the time-bandwidth product is minimised for transform-limited, i.e. unchirped, pulses. In other words, at a given spectral bandwidth the shortest pulse durations are associated with transform-limited pulses, for which case the relation $\Delta \omega \propto (\Delta t)^{-1}$ holds. \par 
The effects on a lase pulse when travelling through a medium can be described via additional terms in the spectral phase: $\varphi(\omega) \to \varphi(\omega) + [\beta(\omega) + i \alpha(\omega)]L$, where $L$ corresponds to the distance travelled in a medium, $\beta(\omega)$ is the propagation coefficient, and $\alpha(\omega)$ the absorption coefficient. The latter term represents an attenuation of the pulse while the former has the effect of introducing chirp (provided the Taylor-expansion of $\beta(\omega)$ has non-zero terms higher than $\beta_1$). Note that depending on the signs of the quantities involved, the chirp due to $\beta(\omega)$ will either increase or compensate the existing chirp on the pulse. Thus, an initially positively chirped pulse will further broaden upon propagation through a normally dispersive medium while a negatively chirped pulse will be compressed. The opposite effect is observed in anomalously dispersive media.

\subsection{Nonlinear Optics and THG}
For most materials, the polarisation response to an electric field of moderate intensity is linear. The peak intensities reached in ultrashort laser pulses are large enough, however, for this approximation to become invalid and nonlinear polarisation components to appear. The resulting polarisation can then be represented as a combination of a linear and a nonlinear term: $\mathbf{P}(\mathbf{E}(t)) = \mathbf{P}^\text{L}(\mathbf{E}(t)) + \mathbf{P}^\text{NL} (\mathbf{E}(t))$, where spatial dependencies have been left implicit. \par 
The linear polarisation is given by the familiar expression $\mathbf{P}^\text{L}(\mathbf{E}(t)) = \epsilon_0 \chi_e \mathbf{E}(t)$, where $\chi_e$ is the (linear) electric susceptibility. The nonlinear term, $\mathbf{P}^\text{NL}$, on the other hand, is less trivial. It can be approximated using a perturbative approach in combination with a contribution to take into account the effects of photoionisation, which is non-perturbative:
\begin{equation}\label{eq:p_nl}
\mathbf{P}^\text{NL}(\mathbf{E}(t)) = \epsilon_0 \sum_{n=2} \chi^{(n)} [\mathbf{E}(t)]^n + \mathbf{P}^\text{ion}(\mathbf{E}(t)),
\end{equation}
where $\chi^{(n)}$ is the $n$-th order electric susceptibility, which is assumed to be a frequency-independent scalar. Note that the series expansion starts at the quadratic term as the linear term is contained in $\mathbf{P}^\text{L}$.  \par 
In centrosymmetric media such as gases all even-order susceptibilities necessarily vanish, so that the dominant term in the expansion is the third-order term. Higher-order contributions can be neglected to a good approximation, so that \eqref{eq:p_nl} becomes 
\begin{equation}\label{eq:p_nl_simpl}
\mathbf{P}^\text{NL}(\mathbf{E}(t)) = \epsilon_0 \chi^{(3)}|\mathbf{E}(t)|^2 \mathbf{E}(t) + \mathbf{P}^\text{ion}(\mathbf{E}(t)).
\end{equation}
As third-order nonlinear effects are commonly known as Kerr effects the first term of \eqref{eq:p_nl_simpl} is known as the Kerr term: $\mathbf{P}^\text{Kerr}(\mathbf{E}(t)) = \chi^{(3)}|\mathbf{E}(t)|^2 \mathbf{E}(t)$. \par 
How the presence of the Kerr terms leads to THG can be seen when considering an oscillating electric field $\mathbf{E}(t) = \hat{E}(t)\cos(\omega t)$, where all spatial dependence and phase terms have been absorbed into the complex envelope function $\tilde{\mathbf{E}}(t)$. The resulting Kerr response is 
\begin{equation}\label{eq:Kerr_response}
\mathbf{P}^\text{Kerr}(\mathbf{E}(t)) = \epsilon_0 \chi^{(3)} \tilde{\mathbf{E}}(t)|\tilde{\mathbf{E}}(t)|^2 \cos^3(\omega t) =\frac{1}{4} \epsilon_0 \chi^{(3)}  \tilde{\mathbf{E}} |\tilde{\mathbf{E}}(t)|^2 \left[  3 \cos(\omega t) + \cos (\omega t) \right].
\end{equation}
The second summand corresponds to a field contribution oscillating at the frequency $3 \omega$. Thus, a pump wave at the fundamental frequency $\omega$ will generate a co-propagating frequency-tripled field. This process can be identified as THG. Note that THG is just once instance of a range of nonlinear optical effects which result in the generation of higher-frequency waves from a driving field, generally known as frequency up-conversion. \par 
The energy of the third harmonic, and by extension the THG efficiency $\eta_THG$, is determined by the value of $\chi^{(3)}$ in a given medium. A further factor to consider, however, is that the third harmonic is generated in-phase with the driving wave at each point in the nonlinear medium. Thus, in dispersive media there will be phase differences between third harmonic contributions generated at different points along the crystal. This can result in destructive interference and thus lower the THG conversion efficiency. \par 
Note that, as \eqref{eq:Kerr_response} suggests, the intensity of the third harmonic is most significantly affected by the intensity of the driving field. This also implies that, since the intensity of the fundamental pulse is highest at the peak of the envelope, THG conversion is more efficient there so that the third harmonic pulse is narrower in time than the input pulse by a factor of $\sim\sqrt{3}$. \par 
The first term of \eqref{eq:Kerr_response}, corresponding to polarisation response at the fundamental frequency is also significant: its effect is a correction term in the refractive index that is proportional to the intensity $I \propto|\tilde{\mathbf{E}}|^2$ of the driving field. This intensity-dependent refractive index (IDRI) produced by the Kerr effect has important implications for the propagation of laser pulses in nonlinear media. These include self-phase modulation (SPM) as well as self-steepening. SPM arises because the intensity of a laser pulse is higher at the peak of its envelope than at the wings, so that the peak will experience a higher refractive index, and therefore phase shift on propagation, than the edges of the pulse. This induces chirp and can result in spectral broadening. Similarly, self-steepening describes the effect of the peak of the pulse experiencing, due to the IDRI, a different group velocity in a dispersive medium than the wings. This results in the leading edge of the pulse or, in the case of anomalous dispersion, the trailing edge, to get steeper on propagation, corresponding to temporal compression and spectral broadening. \par 
After having considered the effects of the Kerr term in the previous paragraphs, we now turn to the second term in \eqref{eq:p_nl}: photoionisation. The highly energetic driving pulse can ionise the atoms or molecules in the nonlinear medium, resulting in the presence of free electrons. This has a two-fold effect on pulse propagation and THG. On the one hand, the pump field is depleted since it must overcome the atomic ionisation potential, so that less energy is available to drive THG. On the other hand, the freed electrons also interact electromagnetically with the fundamental and third harmonic fields, which results in spatio-temporal reshaping of the pulses. Thus, despite ionisation limiting the achievable third harmonic pulse energy, it can aid in further shortening the generated pulses. \par 
The theoretical treatment of $\mathbf{P}^\text{ion}(\mathbf{E}(t))$ is significantly more sophisticated than that of the Kerr term. As is derived in \cite{geissler1999}:
\begin{equation}\label{eq:p_ion}
\frac{\partial}{\partial t} \mathbf{P}^\text{ion}(t, \mathbf{r},z) = \frac{I_p}{\mathbf{E}(t)} \frac{\partial n_e}{\partial t} + \frac{e^2}{m_e^2} \int_{-\infty}^t n_e(t') \mathbf{E}(t')dt',
\end{equation} 
where $n_e(t)$ is the number of free electrons at any point in time, $I_p$ is the ionisation potential, and $e$, $m_e$ represent the electron charge and mass, respectively. The value of $n_e(t)$ can be found from 
\begin{equation}
\frac{\partial n_e}{\partial t} = (n_e - n_0) W(\mathbf{E}(t)),
\end{equation}
where $n_0$ is the initial number of neutral atoms and $W(\mathbf{E}(t))$ is the ionisation rate, which can be calculated from the so-called ADK model of tunnel ionisation or the PPT model, which is more sophisticated but also more computationally expensive. 

-----------------------------------------------------------
\begin{itemize}
\item clean up the maths 
\item cite more/fewer things?!
\end{itemize}

\subsection{The Unidirectional Pulse Propagation Equation}

A variety of different differential equations  can be used to describe the propagation of laser pulses. Many of them are derived under the slowly-varying- envelope assumption (SVEA), which approximates the carrier wave to change significantly more quickly in space and time then the pulse envelope. The SVEA does not hold for ultrashort pulses, however, and hence many commonly used pulse propagation equations are not suitable for the study of few-femtosecond pulses. The unidirectional pulse propagation equation (UPPE) is derived directly from Maxwell's equations, see \cite{kolesik2004}, and therefore holds for ultrashort pulses. It is the differential equation used by \textit{Luna.jl} to simulate laser pulse propagation. This section gives a brief overview of the UPPE and how it is solved in \textit{Luna.jl}. For more details on numerical solutions of the UPPE see the \textit{Luna.jl} documentation, \cite{brahms2023}, and \cite{tani2014}. \par 
The general form of the UPPE for a laser pulse of frequency $\omega$ travelling in the $z$-direction is
\begin{equation}\label{eq:UPPE}
\frac{\partial}{\partial z} \mathbf{E}(\omega, \mathbf{k}_\perp, z)= \mathcal{L}(\omega, \mathbf{k}_\perp, z)\mathbf{E}(\omega, \mathbf{k}_\perp, z) + \frac{i \omega}{N} \mathbf{P}^\text{NL}(\omega, \mathbf{k}_\perp, z),
\end{equation}
where $\mathbf{E}(\omega, \mathbf{k}_\perp, z)$ is the reciprocal-space electric field amplitude in the frequency domain, $\mathbf{k}_\perp$ is the transverse wavevector,  $\mathcal{L}(\omega, \mathbf{k}_\perp, z)$ is a linear operator describing dispersion and absorption, while $\mathbf{P}^\text{NL}(\omega, \mathbf{k}_\perp, z)$ is the nonlinear polarisation in reciprocal space and the frequency domain, and $N$ is a normalisation factor. \par 

----------------------------------------------

As is seen in \eqref{eq:p_nl_simpl} and \eqref{eq:p_ion}, the nonlinear polarisation is naturally calculated from the electric field in the time domain. Thus, numerically solving the UPPE, which lives in the frequency domain, requires a cycle of temporal Fourier transforms into and out of the frequency domain at each step. Similarly, since REAL SPACE, REQUIRES TRANSVERSE SPATIAL TRANSFORM. \par 

SPATIAL TRANSFORM LINKED TO (IDENTICAL WITH??) MODAL DECOMPOSITION??!!

SAY SOMETHING ABOUT REQUIRING DIFFERENT TRANSFORMATIONS; AND ABOUT USING MODAL DECOMPOSITION TO SOLVE ONE SCALAR UPPE PER MODE; THE OVERLAP INTEGRALS ARE A TRANSVERSE SPATIAL TRANSFORM INTO RECIPROCAL SPACE! MENTION THAT USE HANKEL
--------------------------------------------------------
The frequency-domain electric field in real space, with transverse position vector $\mathbf{r}_\perp$, can be decomposed without approximation into a linear combination of orthogonal modes:
\begin{equation}\label{eq:modal_decomp}
\mathbf{E}(\omega, \mathbf{r}_\perp, z) = \sum_j \tilde{E}_j(\omega, z) \boldsymbol{\mathcal{E}}_j(\mathbf{r}_\perp, z),
\end{equation}
where $\tilde{E}_j(\omega, z)$ is the frequency-domain field amplitude of the $j$th mode and $\boldsymbol{\mathcal{E}}_j(\mathbf{r}_\perp, z)$ the transverse field distribution of the mode, which is normalised to ensure that $|\tilde{E}_j(\omega,z)|^2$ represents the spectral power density in the $j$th mode. For pulse propagation without a waveguide, as in the present case of pulses travelling through gas, the modes $\boldsymbol{\mathcal{E}}_j$ correspond to free-space modes. Given an input field $\mathbf{E}(\omega, \mathbf{r}_\perp, z)$ it can be projected onto its modal expansion via the overlap integral
\begin{equation}\label{eq:Ej}
\tilde{E}_j(\omega,z) = \int_\text{all space} [ \boldsymbol{\mathcal{E}}^*_j(\mathbf{r}_\perp, z) \cdot \mathbf{E}(\omega, \mathbf{r}_\perp, z)] d^2 \mathbf{r}_\perp.
\end{equation}
Similarly, $\tilde{P}_j^\text{NL}(\omega,z)$ represents the projection of the polarisation response onto the $j$th mode:
\begin{equation}
\tilde{P}_j^\text{NL}(\omega,z) = \int_\text{all space}[ \boldsymbol{\mathcal{E}}^*(\mathbf{r}_\perp,z) \cdot \mathbf{P}_j^\text{NL}(\omega, \mathbf{r}_\perp, z)] d^2 \mathbf{r}_\perp.
\end{equation}
-----------------------------------------------------------

In Cartesian coordinates, $\mathbf{r}_\perp = (x,y)$, with plane-wave free space modes the integral in \eqref{eq:Ej} corresponds to a two-dimensional spatial Fourier transform. Since the system is radially symmetric, however, it is more efficient to use cylindrical polar coordinates, $\mathbf{r}_\perp = (r, \theta)$, and employ a zeroth-order Hankel transform, $\mathcal{H}_0$. The zeroth-order Hankel transform of a radially symmetric function can be shown to be mathematically equivalent to its two-dimensional Fourier transform in Cartesian coordinates. It is computationally advantageous to work with $\mathcal{H}_0$ in the present case, as this reduces the integral in \eqref{eq:Ej} from one to two dimensions. Thus,
\begin{equation}
\tilde{E}_j(\omega, z) = \mathcal{H}_0 \left[ \boldsymbol{\mathcal{E}}_j(\mathbf{r}_\perp, z) \cdot \mathbf{E}(\omega, \mathbf{r}_\perp, z) \right] \equiv \int_0^\infty  \left[ \boldsymbol{\mathcal{E}}_j(\mathbf{r}_\perp, z) \cdot \mathbf{E}(\omega, \mathbf{r}_\perp, z) \right] J_0(k_r r)r dr,
\end{equation}
where $J_0$ is the zeroth-order Bessel function and $k_r$ is the radial spatial frequency. 


-----------------------------------------------
The linear operator for a mode $j$ is given by 
\begin{equation}\label{eq:Lj}
\mathcal{L}_j(\omega,z) = i \left( \beta_j(\omega, z) - \frac{\omega}{z} \right) - \frac{1}{2} \alpha_j (\omega,z),
\end{equation}
where $\beta_j(\omega,z)$ is the propagation factor of the mode, $v$ is a reference frame velocity (arbitrarily chosen to be the group velocity at the central wavelength $\lambda_0$), and $\alpha_j(\omega,z)$ is the absorption factor.  
-----------------------------------------------------
Modal decomposition can be used to solve the UPPE separately for the field of each mode, with the nonlinear term taking into account potential coupling between different modes:
\begin{equation}
\frac{\partial}{\partial z} \tilde{E}_j(\omega,z) = \mathcal{L}_j (\omega,z) \tilde{E}_j(\omega,z)+ i\frac{\omega}{N_{j}} \tilde{P}_{j}^\text{NL},
\end{equation}
---------------------------------------------------

The nonlinear polarisation response to a given electric field is computed most easily in real space and the time domain: $\mathbf{P}(t, \mathbf{r}_\perp,z) = \mathcal{P}_\text{nl}(\mathbf{E}(t,\mathbf{r}_\perp,z))$, where $t$ is time, $\mathcal{P}_\text{nl}$ is an appropriate operator [RELATE TO Pnl ABOVE?!] and $\mathbf{r}_\perp = (r, \theta)$ represents the transverse spatial coordinate. The real-space-time domain representation of $\mathbf{P}_\text{nl}$ can then be related to the reciprocal-space-frequency domain version via a temporal Fourier transform, $\mathcal{F}$, as well as an appropriate transverse spatial transform, $\mathcal{T}_\perp$:
\begin{equation}
\mathbf{P}(\omega, \mathbf{k}_\perp,z) = \mathcal{F}\left[ \mathcal{T}_\perp \left[ \mathbf{P}(t, \mathbf{r}_\perp,z) \right](t, \mathbf{k}_\perp, z)  \right]. 
\end{equation}
Note that the real-space-time domain field $\mathbf{E}(t, \mathbf{r}_\perp, z)$ can be recovered from $\mathbf{E}(\omega, \mathbf{k}_\perp, z)$ via the appropriate inverse transformations $\mathcal{F}^{-1}$ and $\mathcal{T}^{-1}_\perp$. \par 

Hence, once the operators $\mathcal{L}$ and $\mathcal{P}_\text{nl}$ as well as appropriate definitions of the transverse (reciprocal) space $\mathbf{k}_\perp$ and associated transform $\mathcal{T}_\perp$ are known, the UPPE can be solved as an initial value problem in $z$, using appropriate cycles of transforms at each step. \par 
The choice of $\mathbf{k}_\perp$ and $\mathcal{T}_\perp$ is mostly a matter of the modal expansion of the field in terms of a set of orthogonal spatial modes. These may be waveguide modes or, in the present case, free space modes. 



-------------------------------------------------------------------------------------------------------------------------





\section{Experimental Conditions}
The CFEL-ATTO group generates few-femtosecond UV laser pulses via THG in a gas cell. This is driven by a pulsed near-infrared (NIR) Ti:sapphire laser with a central wavelength of 750nm and a repetition rate of 1kHz. Various pulse-shaping and compression techniques are used to produce NIR pulses with a duration of 5-6fs and a pulse energy of XX$\mu$J. A (variable) fraction of the pump beam is then linearly polarised and focussed into a vacuum chamber for UV generation, resulting in a spatially approximately Gaussian beam with a beam waist of 65$\mu$m at the focus. The chamber contains a fused silica cell, a gas nozzle and a pumping system. Its specifications are central to the THG process and are described in more detail in the following. \par 
The fused silica cell is equipped with a laser machined central 3mm-long channel with a 0.5mm diameter. Pressurised gas is pumped through the channel, with a two-stage differential pumping system on both sides of the central cell ensuring a tight confinement of the gas in this region. As the pump beam passes through the cell and interacts with the pressurised gas, THG takes place which results in UV pulses being generated. The UV components are then separated from the pump beam upon exiting the vacuum chamber using dispersive mirrors. 

-----------------------------
\begin{itemize}
\item include figure?!!!
\end{itemize}



\section{Simulation Results}

%The simulations were carried out with \emph{Luna}, as described above. The simulation parameters were set up to match the experimental parameters are closely as possible. \par 
%The gas cell was modelled as a 3mm long radially symmetric chamber with a 0.5mm diameter. The spatial profile of the input beam was taken to be a Gaussian with a 65$\mu$m waist size focussed at the centre of the cell, based on measurements of the spatial beam profile. The input NIR pulse was fed in based on spectral measurements (FROG) of its temporal profile [SENTENCE MAKES NO SENSE]. The central wavelength was taken to be 730nm, based on NIR input pulse measurements, whereas the pulse length of the input pulse was 5.9fs. The pressure at the cell edges was taken to be 1mbar. The total wavelength window considered ranged from 100nm to 1000nm, with the range 100-360nm considered as the UV component and 600-1000nm as the IR component. The nonlinear effects considered where the Kerr effect as well as ADK ionisation, although the latter was turned off in some cases, in order to isolate the effects of ionisation. The CEO phase of the input beam was set to zero without measurement, as simulations showed little impact of the CEO phase on the simulation output, since the input pulse involves sufficient cycles. \par 
%Note that two different models for the gas density profile were considered. The first one considered only the cell itself, with a simple density gradient constructed based on specifying the central pressure as well as the edge pressure. The second model was more sophisticated and based on simulations produced with the COMSOL Multiphysics software. It took into account not only the gas cell itself but also the geometry of the 7mm glass chip surrounding it, as well as the effects of the differential pumping system. \par 
%The simulations were focussed on investigating parameters that could be experimentally controlled. These were the gas, the central gas pressure, and the input beam power. The gases considered were Argon, Neon, Helium, Krypton, Xenon, Nitrogen, and Nitrous Oxide (N$_2$O). The beam powers considered were 75mW, 150mW, 200mW, 300mW, and 400mW. The simulated central pressures ranged from 0.1bar to 5.1 bar in 0.1bar increments. \par 
%The simulation output included the UV spectrum at the end of the cell (or chip), the total output UV energy, the THG conversion efficiency, the pulse duration of the UV pulse, and the cell (or chip) position at which the UV energy intensity peaked. Additionally, the option to visualise the results in more detail was included, with plots such as: time-domain IR pulse profile, time-domain UV pulse profile, spatiotemporal UV beam evolution, spatiotemporal IR beam evolution, on-axis frequency evolution, UV spectral evolution, on- and off-axis UV and IR intensity profiles, UV and IR energy profiles along z, various spectra. \par 


%NOTE: ADK MODEL DOES NOT INVOLVE MULTIPHOTON IONISATION BUT IS COMPUTATIONALLY LESS EXPENSIVE
%-> inaccurate at low pressures! 

%THIS ENTIRE SECTION IS ESSENTIALLY USELESS!!

\section{Experimental Results}

\section{Discussion}
Comparison with results at times difficult: pressure and energy of output beam hard to measure


Many more factors to be investigated (beam focus position, beam size, higher beam powers...)


Improve simulations by improving COMSOL gas density model (need to use high Mach number model instead, current model does not take into account gas interaction). 

\section{Conclusions}

\section*{Acknowledgments}
I want to thank Josina Hahne for the supervision of the project and Vincent Wanie for his guidance and support. I also want to thank Chris Brahms for providing help with \textit{Luna.jl} as well as Olaf Behnke and Andreas Przystawik for organising the DESY Summer Student programme.  


\section*{References}
\bibliographystyle{osa}
\bibliography{refs.bib}

\end{document}